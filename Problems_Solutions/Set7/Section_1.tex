Consider a bivariate random variable:


\begin{equation}
\nonumber
\mathbi{X}_t=\left[\begin{array}{c} X_{1} \\ X_{2}, \end{array}\right]
\end{equation}

with mean and covariance:

$(\epsilon_{1,t}, \epsilon_{2,t})^T$ are $i.i.d.N(0_2,\Sigma)$, and 

\begin{equation}
\nonumber
E[X] = \left[\begin{array}{c} \alpha_{1} \\  \alpha_{2} \end{array}\right], and \hfill
Cov[X] = \Sigma = \left[\begin{array}{cc} \Sigma_{1,1}  &  \Sigma_{1,2} \\  
                                                                 \Sigma_{2,1}  &  \Sigma_{2,2} \end{array}\right]
                           = \left[\begin{array}{cc} \sigma_{1}^2  &  \rho\sigma_{1}\sigma_{2} \\  
                                                                 \rho\sigma_{1}\sigma_{2}   & \sigma_{2}^2  \end{array}\right],                                      
\end{equation}
where $\sigma_1 = \sqrt{\Sigma_{1,1}}$, $\sigma_2 = \sqrt{\Sigma_{2,2}}$ and $\rho$ is the correlation between $\mathbi{X}_1 and \mathbi{X}_2$.

Conduct the Principal Components Analysis (PCA) of $\mathbi{X}$:

% subsections
\subsection{a} 
\label{section_1_a}
\par
Compute the \textbf{eigenvalues} $\Sigma: \lambda_1 \geq \lambda_2 \geq 0$.

\begin{solution}\hfill\break
Calculating eigenvalues:

\begin{equation}
\label{eq:1.1.a:eigenvalues}
\begin{aligned}
(\sigma_{1}^{2}  - \lambda)(  \sigma_{2}^{2}  - \lambda) -   \rho^2\sigma_{1}^{2} \sigma_{2}^{2} & = 0 \\
\lambda^2 -\lambda(\sigma_{1}^{2}  + \sigma_{2}^{2} ) + (1- \rho^2)\sigma_{1}^{2} \sigma_{2}^{2} & = 0
 \end{aligned}
\end{equation}
\hfill\break
Eigenvalues quadratic equation solution :

\begin{equation}
\begin{aligned}
\Delta & = & (\sigma_{1}^{2}  + \sigma_{2}^{2} )^2 - 4 (1- \rho^2)\sigma_{1}^{2} \sigma_{2}^{2}  \\
          & = & \sigma_{1}^{4}  + \sigma_{2}^{4}  + 2\sigma_{1}^{2} \sigma_{2}^{2}  - (4 - 4 \rho^2)\sigma_{1}^{2}\sigma_{2}^{2} \\
          & = & \sigma_{1}^{4}  + \sigma_{2}^{4}   - 2(1 - 2 \rho^2)\sigma_{1}^{2}\sigma_{2}^{2} \\
 \end{aligned}
\end{equation}

\hfill\break
Computing quadratic equation values :

\begin{equation}
\begin{aligned}
\lambda_1 & = & \frac{(\sigma_{1}^{2}  + \sigma_{2}^{2}) + \sqrt{\Delta} }{2} \\
\lambda_2 & = & \frac{(\sigma_{1}^{2}  + \sigma_{2}^{2}) -\sqrt{\Delta} }{2} 
 \end{aligned}
\end{equation}


\end{solution}
\subsection{b} 
\label{section_1_b}
Compute  the \textbf{eigenvectors} $\gamma_1$, $\gamma_2$:

\begin{equation}
\nonumber
\begin{aligned}
\Sigma\gamma_i & = \lambda_i, \quad i = 1,2 &\\
\gamma_i^{\prime}\gamma_i & = 1, \quad i = 1,2 & \\
\gamma_1^{\prime}\gamma_2 & = 0 &
\end{aligned}
\end{equation}

\begin{solution}

\begin{equation}
\label{eq:1.1.b:eigenvectors}
\begin{aligned}
\left[\begin{array}{cc} \sigma_{1}^2  &  \rho\sigma_{1}\sigma_{2} \\  
                                   \rho\sigma_{1}\sigma_{2}   & \sigma_{2}^2  \end{array}\right] 
\left[\begin{array}{c} \gamma_{i,1} \\ \gamma_{i,2} \end{array}\right] &= \left[\begin{array}{cc} \lambda_i & 0 \\  0 & \lambda_i \end{array}\right] \left[\begin{array}{c} \gamma_{i,1} \\ \gamma_{i,2} \end{array}\right]\\
\left[\begin{array}{cc} \sigma_{1}^2  -\lambda_i &  \rho\sigma_{1}\sigma_{2} \\  
                                   \rho\sigma_{1}\sigma_{2}   & \sigma_{2}^2   -\lambda_i \end{array}\right] 
\left[\begin{array}{c} \gamma_{i,1} \\ \gamma_{i,2} \end{array}\right] & = \left[0\right]
\end{aligned}
\end{equation}


\hfill\break
Computing $\gamma_{i,1} =- \frac{\rho\sigma_{1}\sigma_{2} \gamma_{i,2}}{ \sigma_{1}^2  -\lambda_i }$:
Substituting to second row:

\begin{equation}
\begin{aligned}
-\rho\sigma_{1}\sigma_{2} \frac{\rho\sigma_{1}\sigma_{2} \gamma_{i,2}}{ \sigma_{1}^2  -\lambda_i } 
	+( \sigma_{2}^2   -\lambda_i)\gamma_{i,2} & = 0 \\
-\rho\sigma_{1}\sigma_{2} \rho\sigma_{1}\sigma_{2} \gamma_{i,2}
	+ ( \sigma_{1}^2   -\lambda_i) ( \sigma_{2}^2   -\lambda_i)\gamma_{i,2} & = 0	\\
-\rho^2\sigma_{1}^2\sigma_{2}^2 \gamma_{i,2}
	+ (- \sigma_{1}^2 \lambda_i - \sigma_{2}^2 \lambda_i + \sigma_{1}^2 \sigma_{2}^2  +\lambda_i^2)\gamma_{i,2} & = 0  \\
(\rho^2\sigma_{1}^2\sigma_{2}^2 
	- \sigma_{1}^2 \lambda_i - \sigma_{2}^2 \lambda_i + \sigma_{1}^2 \sigma_{2}^2  +\lambda_i^2)\gamma_{i,2} & = 0 \\
\left[\lambda_i^2 -(\sigma_{1}^2 + \sigma_{2}^2 )\lambda_i  -\rho^2\sigma_{1}^2\sigma_{2}^2 
	+ \sigma_{1}^2 \sigma_{2}^2  \right]\gamma_{i,2} & = 0 \\
\left[\lambda_i^2 -(\sigma_{1}^2 + \sigma_{2}^2 )\lambda_i  + (1-\rho^2)\sigma_{1}^2 \sigma_{2}^2 \right])\gamma_{i,2} & = 0 \\
\end{aligned}
\end{equation}
\hfill\break
From \autoref{eq:1.1.a:eigenvalues} we know that: $\lambda_i^2 -(\sigma_{1}^2 + \sigma_{2}^2 )\lambda_i  + (1-\rho^2)\sigma_{1}^2 \sigma_{2}^2  = 0$ therefore $\gamma_{i,2} = c, c \in \mathbb{C}$. Setting $\gamma_{i,2} = \rho\sigma_{1}\sigma_{2} $  one can compute $\gamma_{i,1} $. 

Using second row from \autoref{eq:1.1.b:eigenvectors} we get : $\gamma_{i,1} =\lambda_i - \sigma_{2}^2 $. 

Finally eigenvectors will be:

\begin{equation}
\begin{aligned}
\left[\begin{array}{c} \lambda_1 - \sigma_{2}^2 \\ \rho\sigma_{1}\sigma_{2} \end{array}\right] ,
\left[\begin{array}{c} \lambda_2 - \sigma_{2}^2 \\ \rho\sigma_{1}\sigma_{2} \end{array}\right] 
\end{aligned}
\end{equation}

\hfill\break


           
\end{solution}
%\subsection{c} 
\label{section_1_c}
Compute $\Gamma_1 = Cov[\mathbi{X}_t,\mathbi{X}_{t-1}]$.

\begin{solution}

\begin{equation}
\begin{aligned}
Cov(\mathbi{X}_t,\mathbi{X}_{t-1}) & = E\left\{[\mathbi{X}_t-E(\mathbi{X}_t)][\mathbi{X}_{t-1}-E(\mathbi{X}_{t-1})]^T\right\} \\
									                 & =\left[\begin{array}{cc} Cov(X_{1,t},X_{1,t-1}) & Cov(X_{1,t},X_{2,t-1}) \\ Cov(X_{2,t},X_{1,t-1}) & Cov(X_{2,t},X_{2,t-1}) \end{array}\right] \\
\end{aligned}
\end{equation}

\begin{equation}
\begin{aligned}
Cov(X_{1,t},X_{1,t-1}) 	& = E\left\{[X_{1,t} - E(X_{1,t})][X_{1,t-1} - E(X_{1,t-1})]\right\} \\
												& = E(X_{1,t}X_{1,t-1}) - E(X_{1,t})E(X_{1,t-1}) \\
												& = E(X_{1,t}X_{1,t-1}) - [E(X_{1,t})]^2 \\
												& = E\left(0.3 \cdot X_{1,t-1} + 0.8 \cdot X_{1,t-1}^2 +  X_{1,t-1}\epsilon_{1,t} \right) - [E(X_{1,t})]^2 \\
												& = 0.3 \cdot E(X_{1,t-1}) + 0.8 \cdot E(X_{1,t-1}^2) +  E(X_{1,t-1}\epsilon_{1,t}) - [E(X_{1,t})]^2 \\
												& = 0.3 \cdot E(X_{1,t-1}) + 0.8 \cdot Var(X_{1,t-1}) + 0.8 \cdot [E(X_{1,t-1})]^2 +  E(X_{1,t-1}) \cdot E(\epsilon_{1,t}) - [E(X_{1,t})]^2 \\
												& = 0.3 \cdot E(X_{1,t}) + 0.8 \cdot Var(X_{1,t}) + 0.8 \cdot [E(X_{1,t})]^2 - [E(X_{1,t})]^2 \\
												& = 0.3 \cdot E(X_{1,t}) + 0.8 \cdot Var(X_{1,t}) - 0.2 \cdot [E(X_{1,t})]^2 \\
												& = 0.3 \cdot \frac{3}{2} + 0.8 \cdot \frac{10}{12} - 0.2 \cdot \left(\frac{3}{2}\right)^2 \\
												& = \frac{20}{3}
\end{aligned}
\end{equation}

\begin{equation}
\begin{aligned}
Cov(X'_{1,t},X'_{1,t-1})  & = E\left\{[X_{1,t} - E(X_{1,t})][X_{1,t-1} - E(X_{1,t-1})]\right\} \\
													& = E(X'_{1,t}X'_{1,t-1})\\
													& = E\left(0.8 \cdot {X'}_{1,t-1}^2 +  X'_{1,t-1}\epsilon_{1,t} \right)\\
													& = 0.8 \cdot E({X'}_{1,t-1}^2) \\
													& = 0.8 \cdot \frac{100}{12} \\
													& = \frac{20}{3}
\end{aligned}
\end{equation}

\begin{equation}
\begin{aligned}
Cov(X_{1,t},X_{2,t-1}) 	& = E\left\{[X_{1,t} - E(X_{1,t})][X_{2,t-1} - E(X_{2,t-1})]\right\} \\
													& = E(X_{1,t}X_{2,t-1})\\
													& = E[(0.54 + 0.64 \cdot X_{1,t-2} + 0.8 \cdot \epsilon_{1,t-1} + \epsilon_{1,t})
																 (0.2 + 0.6 \cdot X_{1,t-2} + 0.4 \cdot X_{2,t-2} + \epsilon_{2,t-1})] \\
						& \qquad \qquad - E(X_{1,t})E(X_{2,t-1}) \\
													& = E(0.102 + 0.324 \cdot X_{1,t-2} + 0.216 \cdot X_{2,t-2} + 0.54 \cdot \epsilon_{2,t-1} \\
						& \qquad \qquad + 0.128 \cdot X_{1,t-2} + 0.384 \cdot X_{1,t-2}^2 + 0.256 \cdot X_{1,t-2}X_{2,t-2} + 0.64 \cdot X_{1,t-2}\epsilon_{2,t-1} \\
						& \qquad \qquad + 0.16 \cdot \epsilon_{1,t-1} + 0.48 \cdot X_{1,t-2}\epsilon_{1,t-1} + 0.32 \cdot X_{2,t-2}\epsilon_{1,t-1} + 0.8 \cdot \epsilon_{1,t-1}\epsilon_{2,t-1} \\
						& \qquad \qquad + 0.2 \cdot \epsilon_{1,t} + 0.6 \cdot X_{1,t-2}\epsilon_{1,t} + 0.4 \cdot X_{2,t-2}\epsilon_{1,t} + \epsilon_{1,t}\epsilon_{2,t-1} \\
						& \qquad \qquad - E(X_{1,t})E(X_{2,t-1}) \\
													& = 0.102 + 0.324 \cdot E(X_{1,t-2}) + 0.216 \cdot E(X_{2,t-2}) + 0.54 \cdot E(\epsilon_{2,t-1}) \\
						& \qquad \qquad + 0.128 \cdot E(X_{1,t-2}) + 0.384 \cdot E(X_{1,t-2}^2) + 0.256 \cdot E(X_{1,t-2}X_{2,t-2}) \\
						& \qquad \qquad + 0.64 \cdot E(X_{1,t-2})E(\epsilon_{2,t-1}) \\
						& \qquad \qquad + 0.16 \cdot E(\epsilon_{1,t-1}) + 0.48 \cdot E(X_{1,t-2})(\epsilon_{1,t-1}) \\
						& \qquad \qquad + 0.32 \cdot E(X_{2,t-2})E(\epsilon_{1,t-1}) + 0.8 \cdot (\epsilon_{1,t-1}\epsilon_{2,t-1}) \\
						& \qquad \qquad + 0.2 \cdot E(\epsilon_{1,t}) + 0.6 \cdot E(X_{1,t-2})E(\epsilon_{1,t}) \\
						& \qquad \qquad + 0.4 \cdot E(X_{2,t-2})E(\epsilon_{1,t}) + E(\epsilon_{1,t})E(\epsilon_{2,t-1}) \\
						& \qquad \qquad - E(X_{1,t})E(X_{2,t}) \\
													& = 0.102 + 0.324 \cdot E(X_{1,t}) + 0.216 \cdot E(X_{2,t})\\
						& \qquad \qquad + 0.128 \cdot E(X_{1,t}) + 0.384 \cdot E(X_{1,t}^2) + 0.256 \cdot E(X_{1,t}X_{2,t}) \\
						& \qquad \qquad + 0.8 \cdot (\epsilon_{1,t}\epsilon_{2,t})\\
						& \qquad \qquad - E(X_{1,t})E(X_{2,t-1}) \\
													& = 0.102 + 0.452 \cdot E(X_{1,t}) + 0.216 \cdot E(X_{2,t})\\
						& \qquad \qquad + 0.384 \cdot Var(X_{1,t}) + 0.384 \cdot \left[E(X_{1,t})\right]^2 \\
						& \qquad \qquad + 0.256 \cdot Cov(X_{1,t}, X_{2,t}) + 0.256 \cdot E(X_{1,t})E(X_{2,t})\\
						& \qquad \qquad + 0.8 \cdot (\epsilon_{1,t}\epsilon_{2,t})\\
						& \qquad \qquad - E(X_{1,t})E(X_{2,t}) \\
													& = 0.102 + 0.452 \cdot E(X_{1,t}) + 0.216 \cdot E(X_{2,t})\\
						& \qquad \qquad + 0.384 \cdot Var(X_{1,t}) + 0.384 \cdot \left[E(X_{1,t})\right]^2 \\
						& \qquad \qquad + 0.256 \cdot Cov(X_{1,t}, X_{2,t}) - 0.744 \cdot E(X_{1,t})E(X_{2,t})\\
						& \qquad \qquad + 0.8 \cdot (\epsilon_{1,t}\epsilon_{2,t})\\
													& = 0.102 + 0.452 \cdot \frac{3}{2} + 0.216 \cdot \frac{11}{6}
														+ 0.384 \cdot \frac{100}{12}+ 0.384 \cdot \left(\frac{3}{2}\right)^2 \\
						& \qquad \qquad + 0.256 \cdot \frac{75}{17} - 0.744 \cdot \frac{3}{2}\frac{11}{6}
														+ 0.8 \cdot -1\\
													& = \frac{2107}{598}
\end{aligned}
\end{equation}

\begin{equation}
\begin{aligned}
Cov(X'_{1,t},X'_{2,t-1}) 	& = E\left\{[X_{1,t} - E(X_{1,t})][X_{2,t-1} - E(X_{2,t-1})]\right\} \\
												& = E(X_{1,t}X_{2,t-1}) - E(X_{1,t})E(X_{2,t-1}) \\
												& = E[(0.64 \cdot X'_{1,t-2} + 0.8 \cdot \epsilon_{1,t-1} + \epsilon_{1,t})
															 ( 0.6 \cdot X'_{1,t-2} + 0.4 \cdot X'_{2,t-2} + \epsilon_{2,t-1})]  \\
												& = E(0.384 \cdot {X'}_{1,t-2}^2 + 0.256 \cdot X'_{1,t-2}X'_{2,t-2} + 0.64 \cdot X_{1,t-2}\epsilon_{2,t-1} \\
					& \qquad \qquad + 0.48 \cdot X'_{1,t-2}\epsilon_{1,t-1} + 0.32 \cdot X'_{2,t-2}\epsilon_{1,t-1} + 0.8 \cdot \epsilon_{1,t-1}\epsilon_{2,t-1} \\
					& \qquad \qquad + 0.6 \cdot X'_{1,t-2}\epsilon_{1,t} + 0.4 \cdot X'_{2,t-2}\epsilon_{1,t} + \epsilon_{1,t}\epsilon_{2,t-1})\\
					%%
												& = 0.384 \cdot E({X'}_{1,t-2}^2) + 0.256 \cdot E(X'_{1,t-2}X'_{2,t-2}) + 0.64 \cdot E(X'_{1,t-2})E(\epsilon_{2,t-1}) \\
					& \qquad \qquad + 0.48 \cdot E(X'_{1,t-2})(\epsilon_{1,t-1}) + 0.32 \cdot E(X'_{2,t-2})E(\epsilon_{1,t-1}) + 0.8 \cdot (\epsilon_{1,t-1}\epsilon_{2,t-1}) \\
					& \qquad \qquad + 0.6 \cdot E(X'_{1,t-2})E(\epsilon_{1,t}) + 0.4 \cdot E(X'_{2,t-2})E(\epsilon_{1,t}) + E(\epsilon_{1,t})E(\epsilon_{2,t-1}) \\
					%% 
												& = 0.384 \cdot E{X'}_{1,t}^2) + 0.256 \cdot E(X'_{1,t}X'_{2,t}) \\
					& \qquad \qquad + 0.8 \cdot (\epsilon_{1,t}\epsilon_{2,t}) \\
												& = 0.384 \cdot Var(X'_{1,t}) + 0.256 \cdot Cov(X'_{1,t}, X_{2,t}) + 0.8 \cdot (\epsilon_{1,t}\epsilon_{2,t})\\
					              & = 0.384 \cdot \frac{100}{12} + 0.256 \cdot \frac{75}{17} + 0.8 \cdot -1\\
											  & = \frac{2107}{598}
\end{aligned}
\end{equation}


\begin{equation}
\begin{aligned}
Cov(X_{2,t},X_{2,t-1}) 	& = E\left\{[X_{2,t} - E(X_{2,t})][X_{2,t-1} - E(X_{2,t-1})]\right\} \\
												& = E(X_{2,t}X_{2,t-1}) - E(X_{2,t})E(X_{2,t-1}) \\
												& = E(X_{2,t}X_{2,t-1}) - [E(X_{2,t})]^2 \\
												& = E\left(0.2 \cdot X_{2,t-1} + 0.6 \cdot X_{1,t-1}X_{2,t-1}
													+ 0.4 \cdot X_{2,t-1}^2 + X_{2,t-1}\epsilon_{2,t} \right) - [E(X_{2,t})]^2 \\
												& = 0.2 \cdot E(X_{2,t-1}) + 0.6 \cdot E(X_{1,t-1}X_{2,t-1}) \\
				         & \qquad + 0.4 \cdot E(X_{2,t-1}^2) + E(X_{2,t-1})E(\epsilon_{2,t}) - [E(X_{2,t})]^2 \\
												& = 0.2 \cdot E(X_{2,t}) + 0.6 \cdot Cov(X_{1,t-1},X_{2,t-1}) + 0.6 \cdot E(X_{1,t-1})E(X_{2,t-1}) \\
								 & \qquad + 0.4 \cdot Var(X_{2,t-1}) + 0.4 \cdot [E(X_{2,t-1})]^2 - [E(X_{2,t})]^2 \\
								        & = 0.2 \cdot E(X_{2,t}) + 0.6 \cdot Cov(X_{1,t},X_{2,t}) + 0.6 \cdot E(X_{1,t})E(X_{2,t}) \\
								 & \qquad + 0.4 \cdot Var(X_{2,t}) - 0.6 \cdot [E(X_{2,t})]^2\\
												& = 0.2 \cdot \frac{11}{6} + 0.6 \cdot \frac{100}{12} + 0.6 \cdot \frac{3}{2}\frac{11}{6}
								          + 0.4 \cdot \frac{1150}{119} - 0.6 \cdot \left(\frac{11}{6}\right)^2\\
												& = \frac{1055}{119}
\end{aligned}
\end{equation}

\begin{equation}
\begin{aligned}
Cov(X'_{2,t},X'_{2,t-1}) 	& = E\left\{[X_{2,t} - E(X_{2,t})][X_{2,t-1} - E(X_{2,t-1})]\right\} \\
												& = E(X_{2,t}X_{2,t-1}) - E(X_{2,t})E(X_{2,t-1}) \\
												& = E(X'_{2,t}X'_{2,t-1}) \\
												& = E\left(0.6 \cdot X'_{1,t-1}X'_{2,t-1}
													+ 0.4 \cdot {X'}_{2,t-1}^2 + X'_{2,t-1}\epsilon_{2,t} \right)\\
												& = 0.6 \cdot E(X_{1,t-1}X_{2,t-1}) + 0.4 \cdot E({X'}_{2,t-1}^2) + E(X'_{2,t-1})E(\epsilon_{2,t})\\
												& = 0.6 \cdot Cov(X'_{1,t-1},X'_{2,t-1}) + 0.4 \cdot Var(X'_{2,t-1})\\
								        & = 0.6 \cdot Cov(X'_{1,t},X'_{2,t}) + 0.4 \cdot Var(X'_{2,t})\\
												& = 0.6 \cdot \frac{100}{12} + 0.4 \cdot \frac{1150}{119}\\
												& = \frac{1055}{119}
\end{aligned}
\end{equation}

\begin{equation}
\begin{aligned}
Cov(X_{2,t},X_{1,t-1}) 	& = E\left\{[X_{2,t} - E(X_{2,t})][X_{1,t-1} - E(X_{1,t-1})]\right\} \\
												& = E(X_{2,t}X_{1,t-1}) - E(X_{2,t})E(X_{1,t-1}) \\
												& = E(0.2 \cdot X_{1,t-1} + 0.6 \cdot X_{1,t-1}^2 + 0.4 \cdot X_{1,t-1}X_{2,t-1} + X_{1,t-1}\epsilon_{2,t-1}) \\
								 & \qquad - E(X_{2,t})E(X_{1,t-1}) \\
												& = 0.2 \cdot E(X_{1,t-1}) + 0.6 \cdot E(X_{1,t-1}^2) \\
								 & \qquad + 0.4 \cdot E(X_{1,t-1}X_{2,t-1}) + E(X_{1,t-1}\epsilon_{2,t-1}) - E(X_{2,t})E(X_{1,t}) \\
												& = 0.2 \cdot E(X_{1,t}) + 0.6 \cdot Var(X_{1,t}) + 0.6 \cdot \left[E(X_{1,t})\right]^2 \\
								 & \qquad + 0.4 \cdot Cov(X_{1,t},X_{2,t}) + 0.4 \cdot E(X_{1,t})E(X_{2,t}) - E(X_{2,t})E(X_{1,t}) \\
								        & = 0.2 \cdot E(X_{1,t}) + 0.6 \cdot Var(X_{1,t}) + 0.6 \cdot \left[E(X_{1,t})\right]^2 \\
								 & \qquad + 0.4 \cdot Cov(X_{1,t},X_{2,t}) - 0.6 \cdot E(X_{1,t})E(X_{2,t})\\
												& = 0.2 \cdot \frac{3}{2} + 0.6 \cdot \frac{2}{5} + 0.6 \cdot \left[\frac{3}{2}\right]^2 \\
								 & \qquad + 0.4 \cdot \frac{35}{17} - 0.6 \cdot \frac{3}{2}\frac{11}{6}
\end{aligned}
\end{equation}


\end{solution}
%\subsection{d} 
\label{section_1_d}
Derive formula for computing $\Gamma_h = Cov[\mathbi{X}_t,\mathbi{X}_{t-h}]$.

\begin{solution}

\begin{equation}
\begin{aligned}
Cov(\mathbi{X}_t,\mathbi{X}_{t-h}) & = E\left\{[\mathbi{X}_t-E(\mathbi{X}_t)][\mathbi{X}_{t-1}-E(\mathbi{X}_{t-h})]^T\right\} \\
									                 & =\left[\begin{array}{cc} Cov(X_{1,t},X_{1,t-h}) & Cov(X_{1,t},X_{2,t-h}) \\ Cov(X_{2,t},X_{1,t-h}) & Cov(X_{2,t},X_{2,t-h}) \end{array}\right] \\
																	 & =\left[\begin{array}{cc} \Gamma_{1,1}(h) & \Gamma_{1,2}(h)  \\ \Gamma_{2.1}(h)  & \Gamma_{2,2}(h)  \end{array}\right] \\
\end{aligned}
\end{equation}

\begin{equation}
\begin{aligned}
Cov(X_{1,t},X_{1,t-h}) 	& = E\left\{[X_{1,t} - E(X_{1,t})][X_{1,t-h} - E(X_{1,t-h})]\right\} \\
												& = E(X_{1,t}X_{1,t-h}) - E(X_{1,t})E(X_{1,t-h}) \\
												& = E(X'_{1,t}X'_{1,t-h})\\
												& = E\left(0.8 \cdot X'_{1,t-1}X'_{1,t-h} +  X'_{1,t-h}\epsilon_{1,t} \right)\\
												& = 0.8 \cdot E(X'_{1,t-1}X'_{1,t-h}) \\
												& = 0.8 \cdot Cov(X'_{1,t-1},X'_{1,t-h}) \\
												& = 0.8 \cdot Cov(X'_{1,t},X'_{1,t-(h-1)}) \\
												& = 0.8 \cdot Cov(X_{1,t},X_{1,t-(h-1)}) \\
												& = 0.8 \cdot \Gamma_{1,1}(h-1)) \\
												& = 0.8^{h-1} \cdot \Gamma_{1,1}(0)) \\
\end{aligned}
\end{equation}

\begin{equation}
\begin{aligned}
Cov(X_{1,t},X_{2,t-h}) 	& = E\left\{[X_{1,t} - E(X_{1,t})][X_{2,t-h} - E(X_{2,t-h})]\right\} \\
												& = E(X_{1,t}X_{2,t-h}) - E(X_{1,t})E(X_{2,t-h}) \\
												& = E(X'_{1,t}X'_{2,t-h})\\
												& = E\left(0.8 \cdot X'_{1,t-1}X'_{2,t-h} +  X'_{2,t-h}\epsilon_{1,t} \right)\\
												& = 0.8 \cdot E(X'_{1,t-1}X'_{2,t-h}) \\
												& = 0.8 \cdot E[X'_{1,t-1}(0.6 \cdot X'_{1,t-(h+1)} + 0.4 \cdot X'_{2,t-(h+1)})] \\
												& = 0.48 \cdot E(X'_{1,t-1}X'_{1,t-(h+1)}) + 0.32 \cdot E(X'_{1,t-1}X'_{2,t-(h+1)}) \\
												& = 0.48 \cdot Cov(X'_{1,t},X'_{1,t-h}) + 0.32 \cdot Cov(X'_{1,t},X'_{2,t-h}) \\
												& = 0.48 \cdot \Gamma_{1,1}(h) + 0.32 \cdot Cov(X'_{1,t},X'_{2,t-h}) \\
												& = \frac{0.48}{0.68} \cdot \Gamma_{1,1}(h) \\
												& = \frac{0.48 \cdot 0.8}{0.68} \cdot \Gamma_{1,1}(h-1)) \\
												& = \frac{0.48 }{0.68}\cdot 0.8^{h-1} \cdot \Gamma_{1,1}(0)) \\
\end{aligned}
\end{equation}

\begin{equation}
\begin{aligned}
Cov(X_{2,t},X_{1,t-h}) 	& = E\left\{[X_{2,t} - E(X_{2,t})][X_{1,t-h} - E(X_{1,t-h})]\right\} \\
												& = E(X_{2,t}X_{1,t-h}) - E(X_{2,t})E(X_{1,t-h}) \\
												& = E(X'_{2,t}X'_{1,t-h})\\
												& = E\left(0.6 \cdot X'_{1,t-1}X'_{1,t-h} + 0.4 \cdot X'_{2,t-1}X'_{1,t-h} + X'_{1,t-h}\epsilon_{2,t} \right)\\
												& = 0.8 \cdot E(X'_{1,t-1}X'_{1,t-h}) + 0.4 \cdot E(X'_{2,t-1}X'_{1,t-h}) \\
												& = 0.8 \cdot Cov(X'_{1,t}X'_{1,t-(h-1)}) + 0.4 \cdot Cov(X'_{2,t}X'_{1,t-(h-1)}) \\
												& = 0.8 \cdot \Gamma_{1,1}(h-1) + 0.4 \cdot \Gamma_{2,1}(h-1) \\
												& = 0.8^{h-1} \cdot \Gamma_{1,1}(0) + 0.4 \cdot \Gamma_{2,1}(h-1) \\
\end{aligned}
\end{equation}

\begin{equation}
\begin{aligned}
Cov(X_{2,t},X_{2,t-h}) 	& = E\left\{[X_{2,t} - E(X_{2,t})][X_{2,t-h} - E(X_{2,t-h})]\right\} \\
												& = E(X_{2,t}X_{2,t-h}) - E(X_{2,t})E(X_{2,t-h}) \\
												& = E(X'_{2,t}X'_{2,t-h})\\
												& = E\left(0.6 \cdot X'_{1,t-1}X'_{2,t-h} + 0.4 \cdot X'_{2,t-1}X'_{2,t-h} + X'_{2,t-h}\epsilon_{2,t} \right)\\
												& = 0.8 \cdot E(X'_{1,t-1}X'_{2,t-h}) + 0.4 \cdot E(X'_{2,t-1}X'_{2,t-h}) \\
												& = 0.8 \cdot Cov(X'_{1,t}X'_{2,t-(h-1)}) + 0.4 \cdot Cov(X'_{2,t}X'_{2,t-(h-1)}) \\
												& = 0.8 \cdot \Gamma_{1,2}(h-1) + 0.4 \cdot \Gamma_{2,2}(h-1) \\
												& = 0.8 \cdot\frac{0.48 \cdot 0.8}{0.68} \cdot \Gamma_{1,1}(h-2) + 0.4 \cdot \Gamma_{2,2}(h-1)\\
												& = \frac{0.48 \cdot 0.8}{0.68} \cdot \Gamma_{1,1}(h-1) + 0.4 \cdot \Gamma_{2,2}(h-1)\\
\end{aligned}
\end{equation}

We can rewrite formula for $\Gamma_h = Cov[\mathbi{X}_t,\mathbi{X}_{t-h}]$ :

\begin{equation}
\begin{aligned}
vec(\Gamma_h)  & =\left[\begin{array}{c} \Gamma_{1,1}(h) 							  \\ \Gamma_{1,2}(h)  \\ \Gamma_{2,1}(h) \\ \Gamma_{2,2}(h)  \end{array}\right] 
							   = A
									\cdot 
									\left[\begin{array}{c} \Gamma_{1,1}(h-1) 							\\ \Gamma_{1,2}(h-1)\\ \Gamma_{2,1}(h-1)\\ \Gamma_{2,2}(h-1)\end{array}\right]
\end{aligned}
\end{equation}

where:
\begin{equation}
              A = \left[\begin{array}{llll} 0.8             						& 0                 & 0                 & 0                \\
									                          0.8 \cdot \frac{0.48}{0.68} & 0                 & 0                 & 0                \\
																						0.8 											  & 0                 & 0.4               & 0                \\ 
																						0.8 \cdot \frac{0.48}{0.68} & 0                 & 0                 & 0.4              \end{array}\right]
\end{equation}

To find generic formula we have to do eigenvalues decomposition of matrix $A$:

\begin{equation}
              A = VDV'.
\end{equation}

Because $A$ is triangular matrix $D$ will be just diagonal of $A$:

\begin{equation}
              D = diag(A) 
							  = \left[\begin{array}{llll} 0.8             						& 0                 & 0                 & 0                \\
									                          0                           & 0                 & 0                 & 0                \\
																						0   											  & 0                 & 0.4               & 0                \\ 
																						0                           & 0                 & 0                 & 0.4              \end{array}\right]
\end{equation}

The generic formula would then become:

\begin{equation}
              vec(\Gamma_h)= VD^hV'vec(\Gamma_{h-1}).
\end{equation}

\end{solution}