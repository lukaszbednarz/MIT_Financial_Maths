\subsection{a}
\label{subsection_4_a}
Under the Gaussian model given $n$, $\mu$, $\sigma^2$ prove that the distribution of $\hat{\sigma}^2$ is:

\begin{equation}
\nonumber
\hat{\sigma}^2 \thicksim \frac{\sigma^2}{n-1} \times \chi_{n-1}^2
\end{equation}

\begin{solution}
We have proven in \ref{sigma_hat_distribution_2} that $\frac{n\hat{\sigma}^2}{\sigma^2} \thicksim \chi_{n-1}^2$ for ML estimator:

\begin{equation}
\nonumber
\hat{\sigma}^2 \thicksim \frac{\sigma^2}{n} \times \chi_{n-1}^2
\end{equation}

The proof will be the same for unbiased estimator and we can write:

\begin{equation}
\nonumber
\hat{\sigma}^2 \thicksim \frac{\sigma^2}{n - 1} \times \chi_{n-1}^2
\end{equation}

From above it follows that:

\begin{equation}
\hat{\sigma}^2 \thicksim \frac{\sigma^2}{n-1} \times \chi_{n-1}^2 \square
\end{equation}


\end{solution}

\subsection{b}
\begin{itemize}
\item Using data for 2008, compute the two-sided 95 \% confidence interval for $\sigma^2$, based on daily log returns.
\item Express the interval in terms of the annualized volatitiliy $(\sqrt{253}\sigma)$. Does the sample annual volatility for any other year fall in the 
confidence interval for 2008? 
\end{itemize}

\begin{solution}
For 2008 the daily variance is $6.677100e-04$. The quantiles for 95\% confidence and 252 d.f. are $q_{0.025} = 209.9227$ and $q_{0.975} = 297.8637$.
Using formula:

\begin{equation}
Pr\left(\hat{\sigma}^2\frac{n-1}{q_{0.025}} \leq \sigma^2 \leq \hat{\sigma}^2\frac{n-1}{q_{0.975}} \right) = 0.975
\end{equation}

we can calculate intervals to be:

\begin{equation}
 5.6490e-4 \leq \sigma^2 \leq 8.0155e-4 \square
\end{equation}

\end{solution}

\begin{solution}
The intervals in terms of annualized volatility are :

\begin{equation}
 0.3780 \leq \sqrt{253}\sigma \leq 0.4503 \square
\end{equation}

None of the other year's volatilities  fall into confidence interval of 2008 volatility.
\end{solution}

\subsection{c}
\label{subsection_4_c}

\begin{itemize}
\item Compute the test statistic $S=S_0$ for testing the daily return variance for 2008 is equal to the daily return variance for 2007.

\item Given the value of the test statistic $S_0$, determine the $\alpha-\textit{level}$ at which the null hypothesis is on the boundary 
of beeing just accepted/rejected.
(This level is called the $P-\textit{value}$ of the test statistic. Reporting a test statistic's $P-\textit{value}$ provides evidence 
concernig for/against the test null hypothesis which can be provided without having to specify $\alpha-\textit{level}$.

\item Repeat the previous two questions for testing the equality of the return variance for 2008 so that for 2006. (Note: the degrees of 
freedom for 2006 are the same as those for 2007 so the same $F$ distribution is applicable).
\end{itemize}

\begin{solution}
The test statistic $S_0$ :

\begin{equation}
S=\frac{\hat{\sigma}_{2008}^2}{\hat{\sigma}_{2007}^2} = 6.5552
\end{equation}

We have to reject null hypothesis that $\frac{\hat{\sigma}_{2008}^2}{\hat{\sigma}_{2007}^2} = 1$.

\end{solution}

\begin{solution}
For the test statistic the value above the $\alpha-\textit{value}$ is 0 down to machine precision of R-package.

\end{solution}


\begin{solution}
Because
\begin{equation}
S=\frac{\hat{\sigma}_{2008}^2}{\hat{\sigma}_{2006}^2} = 16.7709
\end{equation}

We have to reject $H_0$ hypothesis and also the $\alpha-\textit{value}$ will be zero.
\end{solution}